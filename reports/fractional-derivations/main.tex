\documentclass[onecolumn,aps,showpacs,superscriptaddress,footinbib,citeautoscript]{revtex4-1}
%%% General
%\documentclass[twocolumn,aps]{revtex4-1}

\bibliographystyle{apsrev4-1}
\usepackage{graphicx}
%%% Typo
\usepackage[utf8]{inputenc}
\usepackage{csquotes}
\usepackage[american]{babel}
\usepackage[T1]{fontenc}
\usepackage{enumerate}
\usepackage{mdwlist}
%usepackage[activate=normal]{pdfcprot}
%usepackage{bbding}
\usepackage{color}
\usepackage{dsfont}
\frenchspacing

%%% Math
\usepackage{amssymb}
\usepackage{amsmath}
\usepackage{amsfonts}
\usepackage{mathrsfs}

%%% Layout
\usepackage{bm}
\usepackage{dcolumn}
\usepackage{color}
%
\usepackage[colorlinks=true,citecolor=blue]{hyperref}
\hypersetup{colorlinks=true,citecolor=blue,linkcolor=red
,urlcolor=blue}

%%% Definitions
\newcommand{\order}[1]{\mathcal{O}\left({#1}\right)}
\newcommand* {\vek}[1]{{\ensuremath{\bm{\mathrm{#1}}}}}
\newcommand* {\vekc}[1]{{\ensuremath{\bm{\mathcal{#1}}}}}
\newcommand* {\co}[1]{\hat{c}_{#1}}
\newcommand* {\codag}[1]{\hat{c}^\dag_{#1}}
\newcommand* {\n}[1]{(\hat{n}_{#1}-1)}
\newcommand* {\dd}[1]{{\ensuremath{\bm{\mathcal{#1}}}}}
\newcommand* {\kk}{\vek{k}}
\newcommand* {\rr}{\vek{r}}
\newcommand* {\dt}{\delta \tau}
\newcommand* {\bra}[1]{\ensuremath{\langle {#1} |}}
\newcommand* {\ket}[1]{\ensuremath{| {#1} \rangle}}
\newcommand* {\braket}[1]{\ensuremath{\langle {#1} \rangle}}
\newcommand* {\ham}{\mathsf{H}}
\newcommand*{\ee}{\ensuremath{\mathrm{e}}}
\DeclareRobustCommand{\clarify}[1]{\rule{1.8ex}{1.8ex} \emph{#1}}
\newcommand*{\BZR}{\color[rgb]{0.,.5,0.}}
\newcommand{\change}[1]{\textcolor{red}{#1}}
\usepackage{amsmath,amssymb}
\usepackage{color}
\usepackage{graphicx}
%usepackage{bbold}
%\usepackage{feynmf}
\usepackage{natbib}

\newcommand{\citen}[1]{%
  \begingroup
    \romannumeral-`\x % remove space at the beginning of \setcitestyle
    \setcitestyle{numbers}%
    \cite{#1}%
  \endgroup
}

%%% BEGIN DOCUMENT
\def\U{{\bf U}}
\def\T{{\bf T}}
\def\G{{{\bf G}}}
\def\Sig{{{\bf \Sigma}}}
\def\D{{\bf D}}
\def\F{{\bf F}}
\def\I{{\bf I}}
\def\B{{\bf B}}
\def\it{{_a{\bf I}_t}}
\begin{document}
\title{A survay about Fractional Order Calculus.}
\date{\today}

 %%%%%%%%%%%%%%%%%%%%%%%%%%%%%%%%%%%%%%%%%%%%%%%%%%%%%%%
 %=========================================================================
 %%%%%%%%%%%%%%%%%%%%%%%%%%%%%%%%%%%%%%%%%%%%%%%%%%%%%%%

\begin{abstract}
Here we will review derivations of fractional calculus.
\end{abstract}

\pacs{ 73.63.-b, 75.70.Cn, 85.75.-d, 73.43.Qt}
\maketitle

 %%%%%%%%%%%%%%%%%%%%%%%%%%%%%%%%%%%%%%%%%%%%%%%%%%%%%%%
 %=========================================================================
 %%%%%%%%%%%%%%%%%%%%%%%%%%%%%%%%%%%%%%%%%%%%%%%%%%%%%%%

\section{Introduction}
 %======================================================
\section{Fractional order integration}\label{method}
\subsection{Useful relations}
\begin{equation}
 \Gamma(\alpha) = \int_0^\infty t^{\alpha-1}e^{-t}dt
\end{equation}
%
\begin{equation}
 B(\alpha,\beta) = \int_0^1 t^{\alpha-1}(1-t)^{\beta-1}dt
\end{equation}
%
\begin{equation}
 B(\alpha,\beta) = \frac{\Gamma(\alpha)\Gamma(\beta)}{\Gamma(\alpha+\beta)}
\end{equation}
\begin{equation}
 \alpha\Gamma(\alpha)=\Gamma(\alpha+1), \Gamma(0)=\infty, \Gamma(1)=1
\end{equation}
\begin{equation}
 (\alpha+n)\cdots\alpha\Gamma(\alpha)=\Gamma(\alpha+n+1)
\end{equation}
%
\subsection{Cachy formula}
\begin{equation}
  _a\I_t^n[F] = \int_a^t d\tau_1 \int_a^{\tau_1} d\tau_2 \int_a^{\tau_2} d\tau_3 \cdots d\tau_{n-1}\int_a^{\tau_{n-1}} F(\tau_n) d\tau_n 
\end{equation}

\begin{eqnarray}
 _a\I_t^n[F] &=& \int_a^t \underbrace{d\tau_1}_{d\nu} \underbrace{_a\I_{\tau_1}^{n-1}[F]}_{u}  
            = \tau_1 {_aI_{\tau_1}^{n-1}[F]}|_a^t- \int_a^t \tau_1 d\tau_1 {_a\I_{\tau_1}^{n-2}[F]} \nonumber \\
            &=& t \times {_a\I_t^{n-1}[F]} - \int_a^t \tau_1 d\tau_1  {_a\I_{\tau_1}^{n-2}[F]} 
            = \int_a^t t d\tau_1 {_a\I_{\tau_1}^{n-2}[F]} - \int_a^t \tau_1 d\tau_1  {_a\I_{\tau_1}^{n-2}[F]} \nonumber \\
            &=& \int_a^t \underbrace{d\tau_1 (t-\tau_1)}_{d\nu} \underbrace{_a\I_{\tau_1}^{n-2}[F]}_u
\end{eqnarray}
  
\begin{eqnarray}
  _a\I_t^n[F] &=& \underbrace{-\frac{1}{2}(t-\tau_1)^2 {_aI_{\tau_1}^{n-2}[F]}|_a^t}_{=0} + \frac{1}{2}\int_a^t (t-\tau_1)^2 d\tau_1 {_a\I_{\tau_1}^{n-3}[F]} \nonumber \\
  &=& \frac{1}{2}\int_a^t (t-\tau_1)^2 d\tau_1 {_a\I_{\tau_1}^{n-3}[F]} \nonumber\\
  &\vdots& \nonumber \\
  &=& \frac{1}{(n-1)!} \int_a^t (t-\tau_1)^{n-1}F(\tau_1)d\tau_1
\end{eqnarray}
\begin{equation}
  _a\I_t^n[F] = \frac{1}{\Gamma(n)} \int_a^t (t-\tau_1)^{n-1}F(\tau_1)d\tau_1
\end{equation}
\subsection{Arbitrary order integration}
Based on Cachy integration formula for repeated integrations one may extend the integer order integration to arbitrary order integration as follows,
\begin{equation}
 _a\I_t^\alpha[F] = \frac{1}{\Gamma(\alpha)} \int_a^t (t-\tau)^{\alpha-1} F(\tau)d\tau
\end{equation}
\section{Fractional order Differentation}\label{FOD}
%
\subsection{Reimann-Liouvile FOD}
%
\begin{eqnarray}
 ^{\bf RL}_a\D_t^\alpha \F(t) =  \frac{d^m}{dt^m} {_a\I_t^{m-\alpha}[\F]}
                              = \frac{1}{\Gamma(m-\alpha)}\frac{d^m}{dt^m} \int_a^t(t-\tau)^{m-\alpha-1}\F(\tau)d\tau
\end{eqnarray}
where $m-1\preceq\alpha\preceq m$.
\subsection{Caputo FOD}
%
\begin{eqnarray}
^{\bf C}_a\D_t^\alpha \F(t) =  _a\I_t^{m-\alpha}\frac{d^m}{dt^m}[\F]
                              = \frac{1}{\Gamma(m-\alpha)}\int_a^t(t-\tau)^{m-\alpha-1}\frac{d^m}{d\tau^m}\F(\tau)d\tau
\end{eqnarray}
where $m-1\preceq\alpha\preceq m$.
\subsection{Gronvald-Letnikov FOD and FOI}
%
\begin{eqnarray}
^{\bf GL}_a\D_t^\alpha \F(t) = \lim_{h\rightarrow 0} \frac{(1-\hat{\T}_h)^\alpha}{h^\alpha}\F(t)
\end{eqnarray}
Using Tylor expansion of $(1-x)^\alpha$,
\begin{eqnarray}
 (1-\hat{\T}_h)^\alpha &=& \sum_{n=0}^\infty \frac{(-1)^n}{n!} \times \prod_{i=1}^{i=k}(\alpha-i+1) \hat{\T}_h^n \nonumber \\
                       &=&   \sum_{n=0}^\infty \frac{(-1)^n}{n!}  \underbrace{\Gamma(\alpha-n+1)\prod_{i=1}^{i=k}(\alpha-i+1)}_{\Gamma(\alpha+1)}\times\frac{1}{\Gamma(\alpha-n+1)} \hat{\T}_h^n \nonumber\\
                       &=& \sum_{n=0}^\infty (-1)^n \frac{\Gamma(\alpha+1)}{\Gamma(\alpha-n+1)\Gamma(n+1)} \hat{\T}_h^n   =  \sum_{n=0}^\infty (-1)^n 
                       \begin{pmatrix} \alpha  \\ n \end{pmatrix} \hat{\T}_h^n\nonumber  
\end{eqnarray}
Thus the GL FOD becomes,
\begin{equation}
 ^{\bf GL}_a\D_t^\alpha \F(t) =  \frac{1}{h^\alpha}\lim_{h\rightarrow 0} \sum_{n=0}^N (-1)^n 
                       \begin{pmatrix} \alpha  \\ n \end{pmatrix} \F(t-nh)
\end{equation}
with $h=t/N$.
One may define GL integration by just $\alpha\rightarrow -\alpha$,
\begin{eqnarray}
 (1-\hat{\T}_h)^{-\alpha} &=& \sum_{n=0}^\infty \frac{(-1)^n}{n!} \times \prod_{i=1}^{i=k}(-\alpha-i+1) \hat{\T}_h^n \nonumber \\
                       &=&   \sum_{n=0}^\infty \frac{(-1)^n}{n!}  \underbrace{\Gamma(\alpha)\prod_{i=1}^{i=k}(\alpha+i-1)}_{\Gamma(\alpha+n)}\times\frac{1}{\Gamma(\alpha)} \hat{\T}_h^n \nonumber\\
                       &=& \sum_{n=0}^\infty (-1)^n\times(-1)^n \frac{\Gamma(\alpha+1)}{\Gamma(\alpha-n+1)\Gamma(n+1)} \hat{\T}_h^n   \nonumber \\
                       &=&  \sum_{n=0}^\infty (-1)^n 
                       \begin{pmatrix} \alpha+n-1  \\ n \end{pmatrix} \hat{\T}_h^n\nonumber  
\end{eqnarray}
Thus GL FOI becomes,
\begin{equation}
 ^{\bf GL}_a\I_t^\alpha =^{\bf GL}_a\D_t^{-\alpha} \F(t) = \lim_{h\rightarrow 0}h^\alpha\sum_{n=0}^N 
                       \begin{pmatrix} \alpha+n-1  \\ n   \end{pmatrix} \F(t-nh)
\end{equation}
with $h=t/N$.
\subsection{The relation between RL, Caputo  and GL FOD}
%
It is instructive to compare \textbf{Caputo} and {\textbf RL}, for $\F(t)=t^\nu$.\\
\textit{\textbf Reimann-Liouvile}:\\
\begin{eqnarray}
 ^{\bf RL}_a\D_t^\alpha t^\nu &=& \frac{1}{\Gamma(m-\alpha)}\frac{d^m}{dt^m} \int_a^t(t-\tau)^{m-\alpha-1}\tau^\nu d\tau\nonumber \\
                             &=& \frac{1}{\Gamma(m-\alpha)}\underbrace{\int_0^1(1-x)^{m-\alpha-1}x^{\nu+1-1} dx}_{{\bf B}(m-\alpha,\nu+1)} \frac{d^m}{dt^m} t^{m+\nu-\alpha}\nonumber \\ 
                             &=& \frac{1}{\Gamma(m-\alpha)}\frac{\Gamma(m-\alpha)\Gamma(\nu+1)}{\Gamma(m-\alpha+v+1)} \frac{d^m}{dt^m} t^{m+\nu-\alpha}\nonumber \\
                             &=&\frac{\overbrace{\prod_{i=1}^m(i+\nu-\alpha)\Gamma(\nu-\alpha+1)}^{\Gamma(m-\alpha+v+1)}\Gamma(\nu+1)}{\Gamma(\nu-\alpha+1)\Gamma(m-\alpha+v+1)}  t^{\nu-\alpha}\nonumber \\
                             &=& \frac{\Gamma(\nu+1)}{\Gamma(\nu-\alpha+1)}t^{\nu-\alpha}
\end{eqnarray}
\textit{\textbf Caputo}:\\
\begin{eqnarray}
^{\bf C}_a\D_t^\alpha t^\nu &=& \frac{1}{\Gamma(m-\alpha)} \int_a^t(t-\tau)^{m-\alpha-1}\frac{d^m}{d\tau^m}\tau^\nu d\tau {~~\bf \Rightarrow\frac{d^m}{d\tau^m}\tau^\nu=0, ~ if ~ \nu\in N, ~ and ~  \nu<m }\nonumber \\
                            &=& \frac{1}{\Gamma(m-\alpha)} \int_a^t(t-\tau)^{m-\alpha-1}\tau^{\nu-m} d\tau \times \prod_{i=0}^{m-1}(\nu-i) \nonumber \\
                            &=& \frac{1}{\Gamma(m-\alpha)}\underbrace{\int_0^1(1-x)^{m-\alpha-1}x^{\nu-m+1-1} dx}_{{\bf B}(m-\alpha,\nu-m+1)}\times \prod_{i=0}^{m-1}(\nu-i) \times t^{\nu-\alpha}\nonumber \\ 
                            &=& \frac{1}{\Gamma(m-\alpha)}\frac{\Gamma(m-\alpha)\overbrace{{\Gamma(\nu-m+1)\prod_{i=0}^{m-1}(\nu-i)}}^{\Gamma(\alpha+1)}}{\Gamma(\nu-\alpha+1)} \times t^{\nu-\alpha} \nonumber \\
                            &=& \frac{\Gamma(\nu+1)}{\Gamma(\nu-\alpha+1)}t^{\nu-\alpha} \\
\end{eqnarray}
Therefore for \textbf{Caputo} FOD we have,
\begin{equation}
 ^{\bf C}_a\D_t^\alpha t^\nu =\begin{cases}
      0 & \text{if $\nu \in N$ and $\nu<m$}\\
       \frac{\Gamma(\nu+1)}{\Gamma(\nu-\alpha+1)}t^{\nu-\alpha} & \text{otherwise}\\
    \end{cases}  
\end{equation}
with $m-1<\nu<m$.\\
\begin{eqnarray}
 ^{\bf C}_a\D_t^\alpha \F(t) &=& \sum_{\nu=0}^\infty \frac{\F^{(\nu)}(0)}{\nu!}\times ^{\bf C}_a\D_t^\alpha t^\nu \nonumber \\
                             &=& \sum_{\nu=m}^\infty \frac{\F^{(\nu)}(0)}{\nu!}\times \frac{\Gamma(\nu+1)}{\Gamma(\nu-\alpha+1)}t^{\nu-\alpha}\nonumber \\
                             &=& \sum_{\nu=m}^\infty \frac{\F^{(\nu)}(0)}{\Gamma(\nu-\alpha+1)}t^{\nu-\alpha}
\end{eqnarray}
and for \textbf{RL} we have,
\begin{eqnarray}
 ^{\bf RL}_a\D_t^\alpha \F(t) = \sum_{\nu=0}^\infty \frac{\F^{(\nu)}(0)}{\Gamma(\nu-\alpha+1)}t^{\nu-\alpha}
\end{eqnarray}
Finally we have,
\begin{equation}\label{RLvsC1}
 ^{\bf RL}_a\D_t^\alpha \F(t)-^{\bf C}_a\D_t^\alpha \F(t) = \sum_{\nu=0}^{m-1} \frac{\F^{(\nu)}(0)}{\Gamma(\nu-\alpha+1)}t^{\nu-\alpha}
\end{equation}
Taking fractional integration on both sides of Eq.~\ref{RLvsC1}, 
\begin{eqnarray}\label{RLvsC2}
 _a\I_t^\alpha\left[^{\bf RL}_a\D_t^\alpha \F(t)-^{\bf C}_a\D_t^\alpha \F(t)\right] &=& \sum_{\nu=0}^{m-1} \frac{\F^{(\nu)}(0)}{\Gamma(\nu-\alpha+1)} {_a\I_t}^\alpha[t^{\nu-\alpha}] \nonumber \\
 &=& \sum_{\nu=0}^{m-1} \frac{\F^{(\nu)}(0)}{\Gamma(\nu-\alpha+1)} \frac{\Gamma(\nu-\alpha+1)}{\Gamma(\nu+1)} t^\nu \nonumber \\
 &=& \sum_{\nu=0}^{m-1} \frac{\F^{(\nu)}(0)}{\Gamma(\nu+1)} t^\nu
\end{eqnarray}
\section{Fourier and Laplace transformations}
\begin{eqnarray}
 \mathscr{L}\{ {^{\bf RL}_a\D_t^\alpha}\F \}&=&\int_{0}^{\infty} {^{\bf RL}_a\D_t^\alpha} \F(t)e^{-st}dt \nonumber\\
                  &=&\frac{1}{\Gamma(m-\alpha)}\int_{0}^{\infty} e^{-st}dt \overbrace{\frac{d^m}{dt^m}\int_a^t (t-\tau)^{m-\alpha-1}\F(\tau)d\tau}^{\U^{(n)}(t)} \nonumber\\
                  &=& \frac{1}{\Gamma(m-\alpha)}\int_{0}^{\infty} \underbrace{e^{-st}}_{\nu} \underbrace{\U^{(n)}(t)dt}_{du} \nonumber\\
                  &=&  \frac{1}{\Gamma(m-\alpha)}\left[ \U^{(n-1)}(0) + s\int_{0}^{\infty} \underbrace{e^{-st}}_{\nu} \underbrace{\U^{(n-1)}(t)dt}_{du} \right] \nonumber\\
                  &=&  \frac{1}{\Gamma(m-\alpha)}\left[ \U^{(n-1)}(0) + s\U^{(n-2)}(0) + s^2\int_{0}^{\infty} \underbrace{e^{-st}}_{\nu} \underbrace{\U^{(n-2)}(t)dt}_{du} \right] \nonumber\\
                  \vdots \nonumber \\
                  &=&  \frac{1}{\Gamma(m-\alpha)}\left[ \sum_{i=0}^{m-1}\U^{(i)}(0)s^{m-i-1} + s^n\sum_{i=0}^\infty \frac{\F^{(\nu)}(a)}{\nu!}\int_{0}^{\infty} e^{-st}\int_a^t (t-\tau)^{m-\alpha-1}\tau^\nu d\tau \right] \nonumber  \\
                  &=&  \frac{1}{\Gamma(m-\alpha)}\left[ \sum_{i=0}^{n-1}\U^{(i)}(0)s^{m-i-1} + s^m\sum_{\nu=0}^\infty \frac{\F^{(\nu)}(a)}{\nu!}\int_{0}^{\infty} e^{-st}t^{m-\alpha+\nu}dt\underbrace{\int_0^1 (1-x)^{m-\alpha-1}x^{\nu+1-1}dx}_{\B(m-\alpha,\nu+1)} \right] \nonumber  \\
                  &=&  \frac{1}{\Gamma(m-\alpha)}\left[ \sum_{\nu=0}^{m-1}\U^{(i)}(0)s^{m-i-1} + s^m\sum_{\nu=0}^\infty \frac{\F^{(\nu)}(a)}{\nu!}\B(m-\alpha,\nu+1)\underbrace{\int_{0}^{\infty} e^{-st}  (st)^{m-\alpha+\nu}d(st)}_{\Gamma(m-\alpha+\nu+1)}\times s^{\alpha-m-\nu-1} \right] \nonumber  \\
                  &=&  \frac{1}{\Gamma(m-\alpha)}\left[ \sum_{\nu=0}^{m-1}\U^{(i)}(0)s^{m-i-1} + s^m\sum_{\nu=0}^\infty \frac{\F^{(\nu)}(a)}{\nu!}\frac{\Gamma(\nu+1)\Gamma(m-\alpha)}{\Gamma(m-\alpha+\nu+1)}\Gamma(m-\alpha+\nu+1) s^{\alpha-m-\nu-1} \right] \nonumber\\
                  &=&  \frac{1}{\Gamma(m-\alpha)}\sum_{\nu=0}^{m-1}\U^{(i)}(0)s^{m-i-1} + s^\alpha\underbrace{\sum_{\nu=0}^\infty \frac{\F^{(\nu)}(a)}{ s^{\nu+1}}}_{\F(s)} \nonumber
\end{eqnarray}
%
\begin{eqnarray}
 \mathscr{L}\{{^{\bf C}_a\D_t^\alpha}\F\}&=&\int_{0}^{\infty} {^{\bf C}_a\D_t^\alpha} \F(t)e^{-st}dt \nonumber\\
                                         &=&\sum_{\nu=0}^\infty \frac{\F^{(\nu)}(a)}{\nu!}\int_{0}^{\infty} {^{\bf C}_a\D_t^\alpha} t^\nu e^{-st}dt \nonumber\\
                                         &=&\sum_{\nu=m}^\infty \frac{\F^{(\nu)}(a)}{\nu!}\frac{\Gamma(\nu+1)}{\Gamma(\nu-\alpha+1)}\underbrace{\int_{0}^{\infty} (st)^{\nu-\alpha} e^{-st}d(st)}_{\Gamma(\nu-\alpha+1)}\times s^{\alpha-\nu-1} \nonumber\\
                                         &=&s^\alpha\sum_{\nu=m}^\infty \frac{\F^{(\nu)}(a)}{s^{\nu+1}}\nonumber\\
                                         &=&\underbrace{\sum_{\nu=0}^\infty \frac{\F^{(\nu)}(a)}{s^{\nu+1}}}_{\F(s)}- \sum_{\nu=0}^{m-1} \F^{(\nu)}(a) s^{\alpha-\nu-1} \nonumber
\end{eqnarray}
%
\section{Numerical Methods of solutions of Ordinary Differential Equations}
%
\begin{equation}\label{nonlin}
 \dot{y}(t) = \F(y(t),t)
\end{equation}
\subsection{Multi-Linear methods for the solutions of ODEs}
%
To solve Eq.~\ref{nonlin}, by assuming we have the values of $y(t)$ for $t<t_0$, one may integrate both sides of Eq.~\ref{nonlin}, over $t$ from $t_0$ to $t_0+h$. To perform, numerically, the integration $y(t)$  and $\F(t)$ could be approximated by a suitable interpolation function. Obviously, the larger the order of the interpolation will allow a larger choose of $h$. For the interpolation of the right hand side of Eq.~\ref{nonlin}, there is two option for the choose of the 
 interpolation point: (1) $y(t_0+h)$ being expluded from the interpolation reference points (predictor). (2) $y(t_0+h)$ being included in the interpolation reference points (corrector). For the later, the solution of the resulting discrete equations must be proceed by iteration. The best starategy for solving Eq.~\ref{nonlin} is to get an estimation of the $y(t_0+h)$ through a predictor equation, then correcting the estimation through corrector. Thus a general form of the discritized equation proceeds as:
 The predictor equation,
 \begin{equation}
  \sum_{r=0}^{N} a_r y_{n+1-r} = \sum_{r=0}^{N} b_r \F_{n-r}
 \end{equation}
 And for the corrector equation, we have:
 \begin{equation}
  \sum_{r=0}^{N} a_r y_{n+1-r} = \sum_{r=0}^{N} b_r \F_{n+1-r}
 \end{equation}
where $y_i=y(t_i)$ and $\F_i= \F(y(t_i),t_i)$.
%
\begin{equation}
\int_{t_n}^{t_{n+1}} y(\tau)\tau =  \int_{t_n}^{t_{n+1}}\tilde{\F}_s(\tau)d\tau, \textbf {with~s=Pr~or~Corr}
\end{equation}
where 
\begin{equation}
\tilde{\F}_{Pr}(\tau)=\sum_{i=0}^N\F(t_{n-i}) L_{N-i}(\tau)
\end{equation}
and
\begin{equation}
\tilde{\F}_{Corr}(\tau)=\sum_{i=0}^N\F(t_{n+1-i}) L_{N-i}(\tau)
\end{equation}
\begin{equation}
L_i(t)=\frac{\prod_{k=0,k\neq i}^{N}(t - t_k)}{\prod_{k=0,k\neq i}^{N}(t_i - t_k)}
\end{equation}
For the left hand side the simplest case is the Reiman integration, thus Predictor and Corrector equations reads,
\begin{itemize}
\item [Predictor:]
\begin{equation}
y^{(P)}(t_{n+1}) = y(t_n)+\sum_{i=0}^N \F(t_{n-i},y_{n-i})\int_{t_n}^{t_{n+1}} L_{N-i}(\tau)d\tau
\end{equation}
\item [Predictor:]
\begin{equation}
y(t_{n+1}) = y(t_n)+\sum_{i=0}^N \F(t_{n+1-i},y^{(P)}_{n+1-i})\int_{t_n}^{t_{n+1}} L_{N-i}(\tau)d\tau
\end{equation}
\end{itemize}
For a four point approximation, without loss of generality, we may consider $t_{3-i}=-ih$. Thus we have,
\begin{eqnarray} 
L_0(t)&=&\frac{(t+h)(t+2h)(t+3h)}{(0+h)(0+2h)(0+3h)} = \frac{1}{6h^3}(t^3+6ht^2+11h^2t+6h^3) \nonumber \\
L_1(t)&=&\frac{(t-0)(t+2h)(t+3h)}{(-h+0)(-h+2h)(-h+3h)} = -\frac{1}{2h^3}(t^3+5ht^2+6h^2t) \nonumber \\
L_2(t)&=&\frac{(t-0)(t+h)(t+3h)}{(-2h+0)(-2h+h)(-2h+3h)} = \frac{1}{2h^3}(t^3+4ht^2+3h^2t) \nonumber \\
L_3(t)&=&\frac{(t-0)(t+h)(t+2h)}{(-3h+0)(-3h+h)(-3h+2h)} = -\frac{1}{6h^3}(t^3+3ht^2+2h^2t) \nonumber
\end{eqnarray}
 For the corrector,  the integration must be performed  in the domain of $(-h,0)$. Performing the integrations we have,
\begin{equation}
{_{-h}\I_0}\left[ L_0\right] = \frac{3h}{8}~,{_{-h}\I_0}\left[ L_1\right] = \frac{19h}{24}~,{_{-h}\I_0}\left[ L_2\right] = \frac{-5h}{24}~,{_{-h}\I_0}\left[ L_3\right] = \frac{h}{24} 
\end{equation}
For the predictor the integration must performed in the domain $(0,h)$,
\begin{equation}
{_0\I_h}\left[ L_0\right] = \frac{3h}{8}~,{_0\I_h}\left[ L_1\right] = \frac{19h}{24}~,{_0\I_h}\left[ L_2\right] = \frac{-5h}{24}~,{_0\I_h}\left[ L_3\right] = \frac{h}{24} 
\end{equation}
\subsection{Runge-Cutta}
%
\section{Numerical Methods for solving Fractional Order Differential Equations(FODE)}
%
\subsection{Multi linear methods for the solutions of FODEs}
%
\subsection{Runge-Cutta}
%
\section{Higher order finite difference FOD}
%
\subsection{Higher order finite difference for integer order derivatives}
%
\subsection{Higher order finite difference for FOD}
%
\section{Numerical implementation}
%
\section{Transient simulation of electrical circuits(passive parts only)}
%
\subsection{Simulation of parts without FO components}
%
\subsection{Simulation of parts with FO components}
%
\subsection{Implementation of transient algorithem}
%
\subsection{Inclusion of SPICE net lists}
\section{Some applications of FOC in the electrochemistry}

\section{Possible integration into QUCS}

\begin{acknowledgments}
\end{acknowledgments}
\section{Appendix}
 %%%%%%%%%%%%%%%%%%%%%%%%%%%%%%%%%%%%%%%%%%%%%%%%%%%%%%%
 %=========================================================================
 %%%%%%%%%%%%%%%%%%%%%%%%%%%%%%%%%%%%%%%%%%%%%%%%%%%%%%%
 \newpage
\nocite{apsrev41Control}
\bibliographystyle{apsrev4-1}
\bibliography{draft1.bib}

\end{document}



