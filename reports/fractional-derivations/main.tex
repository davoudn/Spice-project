\documentclass[onecolumn,aps,showpacs,superscriptaddress,footinbib,citeautoscript]{revtex4-1}
%%% General
%\documentclass[twocolumn,aps]{revtex4-1}

\bibliographystyle{apsrev4-1}
\usepackage{graphicx}
%%% Typo
\usepackage[utf8]{inputenc}
\usepackage{csquotes}
\usepackage[american]{babel}
\usepackage[T1]{fontenc}
\usepackage{enumerate}
\usepackage{mdwlist}
%usepackage[activate=normal]{pdfcprot}
%usepackage{bbding}
\usepackage{color}
\usepackage{dsfont}
\frenchspacing

%%% Math
\usepackage{amssymb}
\usepackage{amsmath}
\usepackage{amsfonts}
\usepackage{mathrsfs}

%%% Layout
\usepackage{bm}
\usepackage{dcolumn}
\usepackage{color}
%
\usepackage[colorlinks=true,citecolor=blue]{hyperref}
\hypersetup{colorlinks=true,citecolor=blue,linkcolor=red
,urlcolor=blue}

%%% Definitions
\newcommand{\order}[1]{\mathcal{O}\left({#1}\right)}
\newcommand* {\vek}[1]{{\ensuremath{\bm{\mathrm{#1}}}}}
\newcommand* {\vekc}[1]{{\ensuremath{\bm{\mathcal{#1}}}}}
\newcommand* {\co}[1]{\hat{c}_{#1}}
\newcommand* {\codag}[1]{\hat{c}^\dag_{#1}}
\newcommand* {\n}[1]{(\hat{n}_{#1}-1)}
\newcommand* {\dd}[1]{{\ensuremath{\bm{\mathcal{#1}}}}}
\newcommand* {\kk}{\vek{k}}
\newcommand* {\rr}{\vek{r}}
\newcommand* {\dt}{\delta \tau}
\newcommand* {\bra}[1]{\ensuremath{\langle {#1} |}}
\newcommand* {\ket}[1]{\ensuremath{| {#1} \rangle}}
\newcommand* {\braket}[1]{\ensuremath{\langle {#1} \rangle}}
\newcommand* {\ham}{\mathsf{H}}
\newcommand*{\ee}{\ensuremath{\mathrm{e}}}
\DeclareRobustCommand{\clarify}[1]{\rule{1.8ex}{1.8ex} \emph{#1}}
\newcommand*{\BZR}{\color[rgb]{0.,.5,0.}}
\newcommand{\change}[1]{\textcolor{red}{#1}}
\usepackage{amsmath,amssymb}
\usepackage{color}
\usepackage{graphicx}
%usepackage{bbold}
%\usepackage{feynmf}
\usepackage{natbib}

\newcommand{\citen}[1]{%
  \begingroup
    \romannumeral-`\x % remove space at the beginning of \setcitestyle
    \setcitestyle{numbers}%
    \cite{#1}%
  \endgroup
}

%%% BEGIN DOCUMENT
\def\k{{\bf k}}
\def\T{{\bf T}}
\def\G{{{\bf G}}}
\def\Sig{{{\bf \Sigma}}}
\def\D{{\bf D}}
\def\F{{\bf F}}
\def\I{{\bf I}}
\def\it{{_a{\bf I}_t}}
\begin{document}
\title{A survay about Fractional Order Calculus.}
\date{\today}

 %%%%%%%%%%%%%%%%%%%%%%%%%%%%%%%%%%%%%%%%%%%%%%%%%%%%%%%
 %=========================================================================
 %%%%%%%%%%%%%%%%%%%%%%%%%%%%%%%%%%%%%%%%%%%%%%%%%%%%%%%

\begin{abstract}
Here we will review derivations of fractional calculus.
\end{abstract}

\pacs{ 73.63.-b, 75.70.Cn, 85.75.-d, 73.43.Qt}
\maketitle

 %%%%%%%%%%%%%%%%%%%%%%%%%%%%%%%%%%%%%%%%%%%%%%%%%%%%%%%
 %=========================================================================
 %%%%%%%%%%%%%%%%%%%%%%%%%%%%%%%%%%%%%%%%%%%%%%%%%%%%%%%

\section{Introduction}
 %======================================================
\section{Fractional order integration}\label{method}
%
\subsection{Cachy formula}
\begin{equation}
  _a\I_t^n[F] = \int_a^t d\tau_1 \int_a^{\tau_1} d\tau_2 \int_a^{\tau_2} d\tau_3 \cdots d\tau_{n-1}\int_a^{\tau_{n-1}} F(\tau_n) d\tau_n 
\end{equation}

\begin{eqnarray}
 _a\I_t^n[F] &=& \int_a^t \underbrace{d\tau_1}_{d\nu} \underbrace{_a\I_{\tau_1}^{n-1}[F]}_{u}  
            = \tau_1 {_aI_{\tau_1}^{n-1}[F]}|_a^t- \int_a^t \tau_1 d\tau_1 {_a\I_{\tau_1}^{n-2}[F]} \nonumber \\
            &=& t \times {_a\I_t^{n-1}[F]} - \int_a^t \tau_1 d\tau_1  {_a\I_{\tau_1}^{n-2}[F]} 
            = \int_a^t t d\tau_1 {_a\I_{\tau_1}^{n-2}[F]} - \int_a^t \tau_1 d\tau_1  {_a\I_{\tau_1}^{n-2}[F]} \nonumber \\
            &=& \int_a^t \underbrace{d\tau_1 (t-\tau_1)}_{d\nu} \underbrace{_a\I_{\tau_1}^{n-2}[F]}_u
\end{eqnarray}
  
\begin{eqnarray}
  _a\I_t^n[F] &=& \underbrace{-\frac{1}{2}(t-\tau_1)^2 {_aI_{\tau_1}^{n-2}[F]}|_a^t}_{=0} + \frac{1}{2}\int_a^t (t-\tau_1)^2 d\tau_1 {_a\I_{\tau_1}^{n-3}[F]} \nonumber \\
  &=& \frac{1}{2}\int_a^t (t-\tau_1)^2 d\tau_1 {_a\I_{\tau_1}^{n-3}[F]} \nonumber\\
  &\vdots& \nonumber \\
  &=& \frac{1}{(n-1)!} \int_a^t (t-\tau_1)^{n-1}F(\tau_1)d\tau_1
\end{eqnarray}
\begin{equation}
  _a\I_t^n[F] = \frac{1}{\Gamma(n)} \int_a^t (t-\tau_1)^{n-1}F(\tau_1)d\tau_1
\end{equation}
\subsection{Arbitrary order integration}
Based on Cachy integration formula for repeated integrations one may extend the integer order integration to arbitrary order integration as follows,
\begin{equation}
 _a\I_t^\alpha[F] = \frac{1}{\Gamma(\alpha)} \int_a^t (t-\tau)^{\alpha-1} F(\tau)d\tau
\end{equation}

\subsection{General relations}
\begin{equation}
 \Gamma(\alpha) = \int_0^\infty t^{\alpha-1}e^{-t}dt
\end{equation}
%
\begin{equation}
 B(\alpha,\beta) = \int_0^1 t^{\alpha-1}(1-t)^{\beta-1}dt
\end{equation}
%
\begin{equation}
 B(\alpha,\beta) = \frac{\Gamma(\alpha)\Gamma(\beta)}{\Gamma(\alpha+\beta)}
\end{equation}
\begin{equation}
 \alpha\Gamma(\alpha)=\Gamma(\alpha+1), \Gamma(0)=\infty, \Gamma(1)=1
\end{equation}
\begin{equation}
 (\alpha+n)\cdots\alpha\Gamma(\alpha)=\Gamma(\alpha+n+1)
\end{equation}
\section{Fractional order Differentation}\label{FOD}
%
\subsection{Reimann-Liouvile FOD}
%
\begin{eqnarray}
 ^{\bf RL}_a\D_t^\alpha \F(t) =  \frac{d^m}{dt^m} {_a\I_t^{m-\alpha}[\F]}
                              = \frac{1}{\Gamma(m-\alpha)}\int_a^t(t-\tau)^{m-\alpha-1}\F(\tau)d\tau
\end{eqnarray}
where $m-1\preceq\alpha\preceq m$.
\subsection{Caputo FOD}
%
\begin{eqnarray}
^{\bf C}_a\D_t^\alpha \F(t) =  _a\I_t^{m-\alpha}[\F]
                              = \frac{1}{\Gamma(m-\alpha)}\int_a^t(t-\tau)^{m-\alpha-1}\frac{d^m}{d\tau^m}\F(\tau)d\tau
\end{eqnarray}
where $m-1\preceq\alpha\preceq m$.
\subsection{Gronvald-Letnikov FOD and FOI}
%
\begin{eqnarray}
^{\bf GL}_a\D_t^\alpha \F(t) = \lim_{h\rightarrow 0} \frac{(1-\hat{\T}_h)^\alpha}{h^\alpha}\F(t)
\end{eqnarray}
Using Tylor expansion of $(1-x)^\alpha$,
\begin{eqnarray}
 (1-\hat{\T}_h)^\alpha &=& \sum_{n=0}^\infty \frac{(-1)^n}{n!} \times \prod_{i=1}^{i=k}(\alpha-i+1) \hat{\T}_h^n \nonumber \\
                       &=&   \sum_{n=0}^\infty \frac{(-1)^n}{n!}  \underbrace{\Gamma(\alpha-n+1)\prod_{i=1}^{i=k}(\alpha-i+1)}_{\Gamma(\alpha+1)}\times\frac{1}{\Gamma(\alpha-n+1)} \hat{\T}_h^n \nonumber\\
                       &=& \sum_{n=0}^\infty (-1)^n \frac{\Gamma(\alpha+1)}{\Gamma(\alpha-n+1)\Gamma(n+1)} \hat{\T}_h^n   =  \sum_{n=0}^\infty (-1)^n 
                       \begin{pmatrix} \alpha  \\ n \end{pmatrix} \hat{\T}_h^n\nonumber  
\end{eqnarray}
Thus the GL FOD becomes,
\begin{equation}
 ^{\bf GL}_a\D_t^\alpha \F(t) =  \frac{1}{h^\alpha}\lim_{h\rightarrow 0} \sum_{n=0}^N (-1)^n 
                       \begin{pmatrix} \alpha  \\ n \end{pmatrix} \F(t-nh)
\end{equation}
with $h=t/N$.
One may define GL integration by just $\alpha\rightarrow -\alpha$,
\begin{eqnarray}
 (1-\hat{\T}_h)^{-\alpha} &=& \sum_{n=0}^\infty \frac{(-1)^n}{n!} \times \prod_{i=1}^{i=k}(-\alpha-i+1) \hat{\T}_h^n \nonumber \\
                       &=&   \sum_{n=0}^\infty \frac{(-1)^n}{n!}  \underbrace{\Gamma(\alpha)\prod_{i=1}^{i=k}(\alpha+i-1)}_{\Gamma(\alpha+n)}\times\frac{1}{\Gamma(\alpha)} \hat{\T}_h^n \nonumber\\
                       &=& \sum_{n=0}^\infty (-1)^n\times(-1)^n \frac{\Gamma(\alpha+1)}{\Gamma(\alpha-n+1)\Gamma(n+1)} \hat{\T}_h^n   \nonumber \\
                       &=&  \sum_{n=0}^\infty (-1)^n 
                       \begin{pmatrix} \alpha+n-1  \\ n \end{pmatrix} \hat{\T}_h^n\nonumber  
\end{eqnarray}
Thus GL FOI becomes,
\begin{equation}
 ^{\bf GL}_a\I_t^\alpha =^{\bf GL}_a\D_t^{-\alpha} \F(t) = \lim_{h\rightarrow 0}h^\alpha\sum_{n=0}^N (-1)^n 
                       \begin{pmatrix} \alpha  \\ n \end{pmatrix} \F(t-nh)
\end{equation}
with $h=t/N$.
\subsection{The relation between RL, GL, and Caputo FOD}
%
\section{Fourier and Laplace transformations}
%
\section{Numerical Methods of solutions of Ordinary Differential Equations}
%
\subsection{Multi linear methods for the solutions of ODEs}
%
\subsection{Runge-Cutta}
%
\section{Numerical Methods for solving Fractional Order Differential Equations(FODE)}
%
\subsection{Multi linear methods for the solutions of FODEs}
%
\subsection{Runge-Cutta}
%
\section{Higher order finite difference FOD}
%
\subsection{Higher order finite difference for integer order derivatives}
%
\subsection{Higher order finite difference for FOD}
%
\section{Numerical implementation}
%
\section{Transient simulation of electrical circuits(passive parts only)}
%
\subsection{Simulation of parts without FO components}
%
\subsection{Simulation of parts with FO components}
%
\subsection{Implementation of transient algorithem}
%
\subsection{Inclusion of SPICE net lists}
\section{Some applications of FOC in the electrochemistry}

\section{Possible integration into QUCS}

\begin{acknowledgments}
\end{acknowledgments}
\section{Appendix}
 %%%%%%%%%%%%%%%%%%%%%%%%%%%%%%%%%%%%%%%%%%%%%%%%%%%%%%%
 %=========================================================================
 %%%%%%%%%%%%%%%%%%%%%%%%%%%%%%%%%%%%%%%%%%%%%%%%%%%%%%%
 \newpage
\nocite{apsrev41Control}
\bibliographystyle{apsrev4-1}
\bibliography{draft1.bib}

\end{document}



