\documentclass[onecolumn,aps,showpacs,superscriptaddress,footinbib,citeautoscript]{revtex4-1}
%%% General
%\documentclass[twocolumn,aps]{revtex4-1}

\bibliographystyle{apsrev4-1}
\usepackage{graphicx}

\usepackage{tikz}
\usepackage{circuitikz}
%%% Typo
\usepackage[utf8]{inputenc}
\usepackage{csquotes}
\usepackage[american]{babel}
\usepackage[T1]{fontenc}
\usepackage{enumerate}
\usepackage{mdwlist}
%usepackage[activate=normal]{pdfcprot}
%usepackage{bbding}
\usepackage{color}
\usepackage{dsfont}
\frenchspacing

%%% Math
\usepackage{amssymb}
\usepackage{amsmath}
\usepackage{amsfonts}
\usepackage{mathrsfs}

%%% Layout
\usepackage{bm}
\usepackage{dcolumn}
\usepackage{color}
%
\usepackage[colorlinks=true,citecolor=blue]{hyperref}
\hypersetup{colorlinks=true,citecolor=blue,linkcolor=red
,urlcolor=blue}

%%% Definitions
\newcommand{\order}[1]{\mathcal{O}\left({#1}\right)}
\newcommand* {\vek}[1]{{\ensuremath{\bm{\mathrm{#1}}}}}
\newcommand* {\vekc}[1]{{\ensuremath{\bm{\mathcal{#1}}}}}
\newcommand* {\co}[1]{\hat{c}_{#1}}
\newcommand* {\codag}[1]{\hat{c}^\dag_{#1}}
\newcommand* {\n}[1]{(\hat{n}_{#1}-1)}
\newcommand* {\dd}[1]{{\ensuremath{\bm{\mathcal{#1}}}}}
\newcommand* {\kk}{\vek{k}}
\newcommand* {\rr}{\vek{r}}
\newcommand* {\G}{\bf G}
\newcommand* {\dt}{\delta \tau}
\newcommand* {\bra}[1]{\ensuremath{\langle {#1} |}}
\newcommand* {\ket}[1]{\ensuremath{| {#1} \rangle}}
\newcommand* {\braket}[1]{\ensuremath{\langle {#1} \rangle}}
\newcommand* {\ham}{\mathsf{H}}
\newcommand*{\ee}{\ensuremath{\mathrm{e}}}
\DeclareRobustCommand{\clarify}[1]{\rule{1.8ex}{1.8ex} \emph{#1}}
\newcommand*{\BZR}{\color[rgb]{0.,.5,0.}}
\newcommand{\change}[1]{\textcolor{red}{#1}}
\usepackage{amsmath,amssymb}
\usepackage{color}
\usepackage{graphicx}
%usepackage{bbold}
%\usepackage{feynmf}
\usepackage{natbib}

\newcommand{\citen}[1]{%
  \begingroup
    \romannumeral-`\x % remove space at the beginning of \setcitestyle
    \setcitestyle{numbers}%
    \cite{#1}%
  \endgroup
}

%%% BEGIN DOCUMENT
\def\U{{\bf U}}
\def\T{{\bf T}}
\def\G{{{\bf G}}}
\def\Sig{{{\bf \Sigma}}}
\def\D{{\bf D}}
\def\F{{\bf F}}
\def\I{{\bf I}}
\def\B{{\bf B}}
\def\it{{_a{\bf I}_t}}
\begin{document}
\title{A survay about Fractional Order Calculus.}
\date{\today}

 %%%%%%%%%%%%%%%%%%%%%%%%%%%%%%%%%%%%%%%%%%%%%%%%%%%%%%%
 %=========================================================================
 %%%%%%%%%%%%%%%%%%%%%%%%%%%%%%%%%%%%%%%%%%%%%%%%%%%%%%%

\begin{abstract}
Here we will review derivations of fractional calculus.
\end{abstract}

\pacs{ 73.63.-b, 75.70.Cn, 85.75.-d, 73.43.Qt}
\maketitle

 %%%%%%%%%%%%%%%%%%%%%%%%%%%%%%%%%%%%%%%%%%%%%%%%%%%%%%%
 %=========================================================================
 %%%%%%%%%%%%%%%%%%%%%%%%%%%%%%%%%%%%%%%%%%%%%%%%%%%%%%%

\section{Introduction}
 %======================================================
\section{Fractional order integration}\label{method}
\subsection{Useful relations}
\begin{equation}
 \Gamma(\alpha) = \int_0^\infty t^{\alpha-1}e^{-t}dt
\end{equation}
%
\begin{equation}
 B(\alpha,\beta) = \int_0^1 t^{\alpha-1}(1-t)^{\beta-1}dt
\end{equation}
%
\begin{equation}
 B(\alpha,\beta) = \frac{\Gamma(\alpha)\Gamma(\beta)}{\Gamma(\alpha+\beta)}
\end{equation}
\begin{equation}
 \alpha\Gamma(\alpha)=\Gamma(\alpha+1), \Gamma(0)=\infty, \Gamma(1)=1
\end{equation}
\begin{equation}
 (\alpha+n)\cdots\alpha\Gamma(\alpha)=\Gamma(\alpha+n+1)
\end{equation}
%
\subsection{Cachy formula}
\begin{equation}
  _a\I_t^n[F] = \int_a^t d\tau_1 \int_a^{\tau_1} d\tau_2 \int_a^{\tau_2} d\tau_3 \cdots d\tau_{n-1}\int_a^{\tau_{n-1}} F(\tau_n) d\tau_n 
\end{equation}

\begin{eqnarray}
 _a\I_t^n[F] &=& \int_a^t \underbrace{d\tau_1}_{d\nu} \underbrace{_a\I_{\tau_1}^{n-1}[F]}_{u}  
            = \tau_1 {_aI_{\tau_1}^{n-1}[F]}|_a^t- \int_a^t \tau_1 d\tau_1 {_a\I_{\tau_1}^{n-2}[F]} \nonumber \\
            &=& t \times {_a\I_t^{n-1}[F]} - \int_a^t \tau_1 d\tau_1  {_a\I_{\tau_1}^{n-2}[F]} 
            = \int_a^t t d\tau_1 {_a\I_{\tau_1}^{n-2}[F]} - \int_a^t \tau_1 d\tau_1  {_a\I_{\tau_1}^{n-2}[F]} \nonumber \\
            &=& \int_a^t \underbrace{d\tau_1 (t-\tau_1)}_{d\nu} \underbrace{_a\I_{\tau_1}^{n-2}[F]}_u
\end{eqnarray}
  
\begin{eqnarray}
  _a\I_t^n[F] &=& \underbrace{-\frac{1}{2}(t-\tau_1)^2 {_aI_{\tau_1}^{n-2}[F]}|_a^t}_{=0} + \frac{1}{2}\int_a^t (t-\tau_1)^2 d\tau_1 {_a\I_{\tau_1}^{n-3}[F]} \nonumber \\
  &=& \frac{1}{2}\int_a^t (t-\tau_1)^2 d\tau_1 {_a\I_{\tau_1}^{n-3}[F]} \nonumber\\
  &\vdots& \nonumber \\
  &=& \frac{1}{(n-1)!} \int_a^t (t-\tau_1)^{n-1}F(\tau_1)d\tau_1
\end{eqnarray}
\begin{equation}
  _a\I_t^n[F] = \frac{1}{\Gamma(n)} \int_a^t (t-\tau_1)^{n-1}F(\tau_1)d\tau_1
\end{equation}
\subsection{Arbitrary order integration}
Based on Cachy integration formula for repeated integrations one may extend the integer order integration to arbitrary order integration as follows,
\begin{equation}
 _a\I_t^\alpha[F] = \frac{1}{\Gamma(\alpha)} \int_a^t (t-\tau)^{\alpha-1} F(\tau)d\tau
\end{equation}
\section{Fractional order Differentation}\label{FOD}
%
\subsection{Reimann-Liouvile FOD}
%
\begin{eqnarray}
 ^{\bf RL}_a\D_t^\alpha \F(t) =  \frac{d^m}{dt^m} {_a\I_t^{m-\alpha}[\F]}
                              = \frac{1}{\Gamma(m-\alpha)}\frac{d^m}{dt^m} \int_a^t(t-\tau)^{m-\alpha-1}\F(\tau)d\tau
\end{eqnarray}
where $m-1\preceq\alpha\preceq m$.
\subsection{Caputo FOD}
%
\begin{eqnarray}
^{\bf C}_a\D_t^\alpha \F(t) =  _a\I_t^{m-\alpha}\frac{d^m}{dt^m}[\F]
                              = \frac{1}{\Gamma(m-\alpha)}\int_a^t(t-\tau)^{m-\alpha-1}\frac{d^m}{d\tau^m}\F(\tau)d\tau
\end{eqnarray}
where $m-1\preceq\alpha\preceq m$.
\subsection{Gronvald-Letnikov FOD and FOI}
%
\begin{eqnarray}
^{\bf GL}_a\D_t^\alpha \F(t) = \lim_{h\rightarrow 0} \frac{(1-\hat{\T}_h)^\alpha}{h^\alpha}\F(t)
\end{eqnarray}
Using Tylor expansion of $(1-x)^\alpha$,
\begin{eqnarray}
 (1-\hat{\T}_h)^\alpha &=& \sum_{n=0}^\infty \frac{(-1)^n}{n!} \times \prod_{i=1}^{i=k}(\alpha-i+1) \hat{\T}_h^n \nonumber \\
                       &=&   \sum_{n=0}^\infty \frac{(-1)^n}{n!}  \underbrace{\Gamma(\alpha-n+1)\prod_{i=1}^{i=k}(\alpha-i+1)}_{\Gamma(\alpha+1)}\times\frac{1}{\Gamma(\alpha-n+1)} \hat{\T}_h^n \nonumber\\
                       &=& \sum_{n=0}^\infty (-1)^n \frac{\Gamma(\alpha+1)}{\Gamma(\alpha-n+1)\Gamma(n+1)} \hat{\T}_h^n   =  \sum_{n=0}^\infty (-1)^n 
                       \begin{pmatrix} \alpha  \\ n \end{pmatrix} \hat{\T}_h^n\nonumber  
\end{eqnarray}
Thus the GL FOD becomes,
\begin{equation}
 ^{\bf GL}_a\D_t^\alpha \F(t) =  \frac{1}{h^\alpha}\lim_{h\rightarrow 0} \sum_{n=0}^N (-1)^n 
                       \begin{pmatrix} \alpha  \\ n \end{pmatrix} \F(t-nh)
\end{equation}
with $h=t/N$.
One may define GL integration by just $\alpha\rightarrow -\alpha$,
\begin{eqnarray}
 (1-\hat{\T}_h)^{-\alpha} &=& \sum_{n=0}^\infty \frac{(-1)^n}{n!} \times \prod_{i=1}^{i=k}(-\alpha-i+1) \hat{\T}_h^n \nonumber \\
                       &=&   \sum_{n=0}^\infty \frac{(-1)^n}{n!}  \underbrace{\Gamma(\alpha)\prod_{i=1}^{i=k}(\alpha+i-1)}_{\Gamma(\alpha+n)}\times\frac{1}{\Gamma(\alpha)} \hat{\T}_h^n \nonumber\\
                       &=& \sum_{n=0}^\infty (-1)^n\times(-1)^n \frac{\Gamma(\alpha+1)}{\Gamma(\alpha-n+1)\Gamma(n+1)} \hat{\T}_h^n   \nonumber \\
                       &=&  \sum_{n=0}^\infty (-1)^n 
                       \begin{pmatrix} \alpha+n-1  \\ n \end{pmatrix} \hat{\T}_h^n\nonumber  
\end{eqnarray}
Thus GL FOI becomes,
\begin{equation}
 ^{\bf GL}_a\I_t^\alpha =^{\bf GL}_a\D_t^{-\alpha} \F(t) = \lim_{h\rightarrow 0}h^\alpha\sum_{n=0}^N 
                       \begin{pmatrix} \alpha+n-1  \\ n   \end{pmatrix} \F(t-nh)
\end{equation}
with $h=t/N$.
\subsection{The relation between RL, Caputo  and GL FOD}
%
It is instructive to compare \textbf{Caputo} and {\textbf RL}, for $\F(t)=t^\nu$.\\
\textit{\textbf Reimann-Liouvile}:\\
\begin{eqnarray}
 ^{\bf RL}_a\D_t^\alpha t^\nu &=& \frac{1}{\Gamma(m-\alpha)}\frac{d^m}{dt^m} \int_a^t(t-\tau)^{m-\alpha-1}\tau^\nu d\tau\nonumber \\
                             &=& \frac{1}{\Gamma(m-\alpha)}\underbrace{\int_0^1(1-x)^{m-\alpha-1}x^{\nu+1-1} dx}_{{\bf B}(m-\alpha,\nu+1)} \frac{d^m}{dt^m} t^{m+\nu-\alpha}\nonumber \\ 
                             &=& \frac{1}{\Gamma(m-\alpha)}\frac{\Gamma(m-\alpha)\Gamma(\nu+1)}{\Gamma(m-\alpha+v+1)} \frac{d^m}{dt^m} t^{m+\nu-\alpha}\nonumber \\
                             &=&\frac{\overbrace{\prod_{i=1}^m(i+\nu-\alpha)\Gamma(\nu-\alpha+1)}^{\Gamma(m-\alpha+v+1)}\Gamma(\nu+1)}{\Gamma(\nu-\alpha+1)\Gamma(m-\alpha+v+1)}  t^{\nu-\alpha}\nonumber \\
                             &=& \frac{\Gamma(\nu+1)}{\Gamma(\nu-\alpha+1)}t^{\nu-\alpha}
\end{eqnarray}
\textit{\textbf Caputo}:\\
\begin{eqnarray}
^{\bf C}_a\D_t^\alpha t^\nu &=& \frac{1}{\Gamma(m-\alpha)} \int_a^t(t-\tau)^{m-\alpha-1}\frac{d^m}{d\tau^m}\tau^\nu d\tau {~~\bf \Rightarrow\frac{d^m}{d\tau^m}\tau^\nu=0, ~ if ~ \nu\in N, ~ and ~  \nu<m }\nonumber \\
                            &=& \frac{1}{\Gamma(m-\alpha)} \int_a^t(t-\tau)^{m-\alpha-1}\tau^{\nu-m} d\tau \times \prod_{i=0}^{m-1}(\nu-i) \nonumber \\
                            &=& \frac{1}{\Gamma(m-\alpha)}\underbrace{\int_0^1(1-x)^{m-\alpha-1}x^{\nu-m+1-1} dx}_{{\bf B}(m-\alpha,\nu-m+1)}\times \prod_{i=0}^{m-1}(\nu-i) \times t^{\nu-\alpha}\nonumber \\ 
                            &=& \frac{1}{\Gamma(m-\alpha)}\frac{\Gamma(m-\alpha)\overbrace{{\Gamma(\nu-m+1)\prod_{i=0}^{m-1}(\nu-i)}}^{\Gamma(\alpha+1)}}{\Gamma(\nu-\alpha+1)} \times t^{\nu-\alpha} \nonumber \\
                            &=& \frac{\Gamma(\nu+1)}{\Gamma(\nu-\alpha+1)}t^{\nu-\alpha} \\
\end{eqnarray}
Therefore for \textbf{Caputo} FOD we have,
\begin{equation}
 ^{\bf C}_a\D_t^\alpha t^\nu =\begin{cases}
      0 & \text{if $\nu \in N$ and $\nu<m$}\\
       \frac{\Gamma(\nu+1)}{\Gamma(\nu-\alpha+1)}t^{\nu-\alpha} & \text{otherwise}\\
    \end{cases}  
\end{equation}
with $m-1<\nu<m$.\\
\begin{eqnarray}
 ^{\bf C}_a\D_t^\alpha \F(t) &=& \sum_{\nu=0}^\infty \frac{\F^{(\nu)}(0)}{\nu!}\times ^{\bf C}_a\D_t^\alpha t^\nu \nonumber \\
                             &=& \sum_{\nu=m}^\infty \frac{\F^{(\nu)}(0)}{\nu!}\times \frac{\Gamma(\nu+1)}{\Gamma(\nu-\alpha+1)}t^{\nu-\alpha}\nonumber \\
                             &=& \sum_{\nu=m}^\infty \frac{\F^{(\nu)}(0)}{\Gamma(\nu-\alpha+1)}t^{\nu-\alpha}
\end{eqnarray}
and for \textbf{RL} we have,
\begin{eqnarray}
 ^{\bf RL}_a\D_t^\alpha \F(t) = \sum_{\nu=0}^\infty \frac{\F^{(\nu)}(0)}{\Gamma(\nu-\alpha+1)}t^{\nu-\alpha}
\end{eqnarray}
Finally we have,
\begin{equation}\label{RLvsC1}
 ^{\bf RL}_a\D_t^\alpha \F(t)-^{\bf C}_a\D_t^\alpha \F(t) = \sum_{\nu=0}^{m-1} \frac{\F^{(\nu)}(0)}{\Gamma(\nu-\alpha+1)}t^{\nu-\alpha}
\end{equation}
Taking fractional integration on both sides of Eq.~\ref{RLvsC1}, 
\begin{eqnarray}\label{RLvsC2}
 _a\I_t^\alpha\left[^{\bf RL}_a\D_t^\alpha \F(t)-^{\bf C}_a\D_t^\alpha \F(t)\right] &=& \sum_{\nu=0}^{m-1} \frac{\F^{(\nu)}(0)}{\Gamma(\nu-\alpha+1)} {_a\I_t}^\alpha[t^{\nu-\alpha}] \nonumber \\
 &=& \sum_{\nu=0}^{m-1} \frac{\F^{(\nu)}(0)}{\Gamma(\nu-\alpha+1)} \frac{\Gamma(\nu-\alpha+1)}{\Gamma(\nu+1)} t^\nu \nonumber \\
 &=& \sum_{\nu=0}^{m-1} \frac{\F^{(\nu)}(0)}{\Gamma(\nu+1)} t^\nu
\end{eqnarray}
\section{Fourier and Laplace transformations}
\begin{eqnarray}
 \mathscr{L}\{ {^{\bf RL}_a\D_t^\alpha}\F \}&=&\int_{0}^{\infty} {^{\bf RL}_a\D_t^\alpha} \F(t)e^{-st}dt \nonumber\\
                  &=&\frac{1}{\Gamma(m-\alpha)}\int_{0}^{\infty} e^{-st}dt \overbrace{\frac{d^m}{dt^m}\int_a^t (t-\tau)^{m-\alpha-1}\F(\tau)d\tau}^{\U^{(m)}(t)} \nonumber\\
                  &=& \frac{1}{\Gamma(m-\alpha)}\int_{0}^{\infty} \underbrace{e^{-st}}_{\nu} \underbrace{\U^{(m)}(t)dt}_{du} \nonumber\\
                  &=&  \frac{1}{\Gamma(m-\alpha)}\left[ \U^{(m-1)}(0) + s\int_{0}^{\infty} \underbrace{e^{-st}}_{\nu} \underbrace{\U^{(m-1)}(t)dt}_{du} \right] \nonumber\\
                  &=&  \frac{1}{\Gamma(m-\alpha)}\left[ \U^{(m-1)}(0) + s\U^{(m-2)}(0) + s^2\int_{0}^{\infty} \underbrace{e^{-st}}_{\nu} \underbrace{\U^{(m-2)}(t)dt}_{du} \right] \nonumber\\
                  \vdots \nonumber \\
                  &=&  \frac{1}{\Gamma(m-\alpha)}\left[ \sum_{\nu=0}^{m-1}\U^{(\nu)}(0)s^{m-\nu-1} + s^m\sum_{\nu=0}^\infty \frac{\F^{(\nu)}(a)}{\nu!}\int_{0}^{\infty} e^{-st}\int_a^t (t-\tau)^{m-\alpha-1}\tau^\nu d\tau \right] \nonumber  \\
                  &=&  \frac{1}{\Gamma(m-\alpha)}\left[ \sum_{\nu=0}^{m-1}\U^{(\nu)}(0)s^{m-\nu-1} + s^m\sum_{\nu=0}^\infty \frac{\F^{(\nu)}(a)}{\nu!}\int_{0}^{\infty} e^{-st}t^{m-\alpha+\nu}dt\underbrace{\int_0^1 (1-x)^{m-\alpha-1}x^{\nu+1-1}dx}_{\B(m-\alpha,\nu+1)} \right] \nonumber  \\
                  &=&  \frac{1}{\Gamma(m-\alpha)}\left[ \sum_{\nu=0}^{m-1}\U^{(\nu)}(0)s^{m-\nu-1} + s^m\sum_{\nu=0}^\infty \frac{\F^{(\nu)}(a)}{\nu!}\B(m-\alpha,\nu+1)\underbrace{\int_{0}^{\infty} e^{-st}  (st)^{m-\alpha+\nu}d(st)}_{\Gamma(m-\alpha+\nu+1)}\times s^{\alpha-m-\nu-1} \right] \nonumber  \\
                  &=&  \frac{1}{\Gamma(m-\alpha)}\left[ \sum_{\nu=0}^{m-1}\U^{(\nu)}(0)s^{m-\nu-1} + s^m\sum_{\nu=0}^\infty \frac{\F^{(\nu)}(a)}{\nu!}\frac{\Gamma(\nu+1)\Gamma(m-\alpha)}{\Gamma(m-\alpha+\nu+1)}\Gamma(m-\alpha+\nu+1) s^{\alpha-m-\nu-1} \right] \nonumber\\
                  &=&  \frac{1}{\Gamma(m-\alpha)}\sum_{\nu=0}^{m-1}\U^{(\nu)}(0)s^{m-\nu-1} + s^\alpha\underbrace{\sum_{\nu=0}^\infty \frac{\F^{(\nu)}(a)}{ s^{\nu+1}}}_{\F(s)} \nonumber
\end{eqnarray}
%
\begin{eqnarray}
 \mathscr{L}\{{^{\bf C}_a\D_t^\alpha}\F\}&=&\int_{0}^{\infty} {^{\bf C}_a\D_t^\alpha} \F(t)e^{-st}dt \nonumber\\
                                         &=&\sum_{\nu=0}^\infty \frac{\F^{(\nu)}(a)}{\nu!}\int_{0}^{\infty} {^{\bf C}_a\D_t^\alpha} t^\nu e^{-st}dt \nonumber\\
                                         &=&\sum_{\nu=m}^\infty \frac{\F^{(\nu)}(a)}{\nu!}\frac{\Gamma(\nu+1)}{\Gamma(\nu-\alpha+1)}\underbrace{\int_{0}^{\infty} (st)^{\nu-\alpha} e^{-st}d(st)}_{\Gamma(\nu-\alpha+1)}\times s^{\alpha-\nu-1} \nonumber\\
                                         &=&s^\alpha\sum_{\nu=m}^\infty \frac{\F^{(\nu)}(a)}{s^{\nu+1}}\nonumber\\
                                         &=&\underbrace{\sum_{\nu=0}^\infty \frac{\F^{(\nu)}(a)}{s^{\nu+1}}}_{\F(s)}- \sum_{\nu=0}^{m-1} \F^{(\nu)}(a) s^{\alpha-\nu-1} \nonumber
\end{eqnarray}
%
\section{Series expansion solution of FODEs}
The simplest example is 
\begin{equation}\label{fode-caputo}
^{\bf C}_a\D_t^\alpha \F(t) + \lambda \F(t) = 0
\end{equation}
Suppose, $\F(t) =\sum_{\nu=0}^\infty a_\nu t^{\nu \alpha}$.
%
\begin{eqnarray}
 \sum_{\nu=0}^\infty  a_\nu {^{\bf C}_a\D_t^\alpha}  t^{\nu \alpha} + \lambda\sum_{\nu=0}^\infty a_\nu t^{\nu t} =
 \sum_{\nu=0}^\infty \left[ a_{\nu+1} \frac{\Gamma((\nu+1)\alpha+1)}{\Gamma(\nu\alpha+1)} + \lambda a_\nu  \right]t^{\nu \alpha} =0
\end{eqnarray}
\begin{equation}
  a_{\nu+1}  = -\lambda \frac{\Gamma(\nu\alpha+1)}{\Gamma((\nu+1)\alpha+1)}a_\nu 
\end{equation}
\begin{eqnarray}
 a_{\nu+1}  &=& -\lambda \frac{\Gamma(\nu\alpha+1)}{\Gamma((\nu+1)\alpha+1)}a_\nu \nonumber \\
 a_{\nu+1}  &=& -\lambda \frac{\Gamma(\nu\alpha+1)}{\Gamma((\nu+1)\alpha+1)}\times-\lambda\frac{\Gamma(\nu\alpha+1)}{\Gamma((\nu+1)\alpha+1)}a_{\nu-1} \nonumber \\
 a_{\nu+1}  &=& -\lambda \frac{\Gamma(\nu\alpha+1)}{\Gamma((\nu+1)\alpha+1)}\times-\lambda\frac{\Gamma(\nu\alpha+1)}{\Gamma((\nu+1)\alpha+1)}a_{\nu-1}\cdots \times -\lambda\frac{\Gamma(1)}{\Gamma(\alpha+1)}a_{0}
\end{eqnarray}
\begin{equation}
 a_{\nu} = \frac{(-\lambda)^\nu}{\Gamma(\nu\alpha+1)}a_0
\end{equation}
thus the solution of Eq.~\ref{fode-caputo} becomes,
\begin{equation}
 \F(t)=\F(0) {\bf E}_{\alpha}(-\lambda t^\alpha)
\end{equation}
The $\F(0)$ fixes the initial condition for Eq.~\ref{fode-caputo}.
the function,
\begin{equation}
 {\bf E}_{\alpha}(x)=\sum_{\nu=0}^\infty \frac{x^\nu}{\Gamma(\nu\alpha+1)}
\end{equation}
is dubed the one parameter Mittag-Leffer function.
One may also employ the Laplace transform for solving Eq.~\ref{fode-caputo},
\begin{eqnarray}
 \mathscr{L}\{ {^{\bf C}_a\D_t^\alpha} \F(t) + \lambda \F(t) \} &=& 0 \nonumber \\
 -s^{1-\alpha}\F(0) + s^\alpha\F(s) + \lambda\F(s) &=& 0
\end{eqnarray}
\begin{equation}
 \F(s) = \F(0)\frac{s^{1-\alpha}}{\lambda+s^\alpha}
\end{equation}
Thus the solution of Eq.~\ref{fode-caputo}, becomes,
\begin{equation}
 \F(t)= \mathscr{L}^{-1}\{ \F(s) \} = \F(0) \int \frac{s^{1-\alpha}}{\lambda+s^\alpha} ds
\end{equation}
therefore one may infer, 
\begin{equation}
  \int \frac{s^{1-\alpha}}{\lambda+s^\alpha}ds = {\bf E}_{\alpha}(-\lambda t^\alpha)
\end{equation}
This means, as long as the $\Re[\lambda] <0$, $\lim_{t\rightarrow+\infty}\F(t)\rightarrow 0$, otherwise if
  $\Re[\lambda] =0$ it oscillates permanently and if $\Re[\lambda] > 0$ the solution diverges.
\section{Numerical Methods of solutions of Ordinary Differential Equations}
%
\begin{equation}\label{nonlin}
 \dot{y}(t) = \F(y(t),t)
\end{equation}
\subsection{Multi-Linear methods for the solutions of ODEs}
%
To solve Eq.~\ref{nonlin}, by assuming we have the values of $y(t)$ for $t<t_0$, one may integrate both sides of Eq.~\ref{nonlin}, over $t$ from $t_0$ to $t_0+h$. To perform, numerically, the integration $y(t)$  and $\F(t)$ could be approximated by a suitable interpolation function. Obviously, the larger the order of the interpolation will allow a larger choose of $h$. For the interpolation of the right hand side of Eq.~\ref{nonlin}, there is two option for the choose of the 
 interpolation point: (1) $y(t_0+h)$ being expluded from the interpolation reference points (predictor). (2) $y(t_0+h)$ being included in the interpolation reference points (corrector). For the later, the solution of the resulting discrete equations must be proceed by iteration. The best starategy for solving Eq.~\ref{nonlin} is to get an estimation of the $y(t_0+h)$ through a predictor equation, then correcting the estimation through corrector. Thus a general form of the discritized equation proceeds as:
 The predictor equation,
 \begin{equation}
  \sum_{r=0}^{N} a_r y_{n+1-r} = \sum_{r=0}^{N} b_r \F_{n-r}
 \end{equation}
 And for the corrector equation, we have:
 \begin{equation}
  \sum_{r=0}^{N} a_r y_{n+1-r} = \sum_{r=0}^{N} b_r \F_{n+1-r}
 \end{equation}
where $y_i=y(t_i)$ and $\F_i= \F(y(t_i),t_i)$.
%
\begin{equation}
\int_{t_n}^{t_{n+1}} y(\tau)\tau =  \int_{t_n}^{t_{n+1}}\tilde{\F}_s(\tau)d\tau, \textbf {with~s=Pr~or~Corr}
\end{equation}
where 
\begin{equation}
\tilde{\F}_{Pr}(\tau)=\sum_{i=0}^N\F(t_{n-i}) L_{N-i}(\tau)
\end{equation}
and
\begin{equation}
\tilde{\F}_{Corr}(\tau)=\sum_{i=0}^N\F(t_{n+1-i}) L_{N-i}(\tau)
\end{equation}
\begin{equation}
L_i(t)=\frac{\prod_{k=0,k\neq i}^{N}(t - t_k)}{\prod_{k=0,k\neq i}^{N}(t_i - t_k)}
\end{equation}
For the left hand side the simplest case is the Reiman integration, thus Predictor and Corrector equations reads,
\begin{itemize}
\item [Predictor:]
\begin{equation}
y^{(P)}(t_{n+1}) = y(t_n)+\sum_{i=0}^N \F(t_{n-i},y_{n-i})\int_{t_n}^{t_{n+1}} L_{N-i}(\tau)d\tau
\end{equation}
\item [Predictor:]
\begin{equation}
y(t_{n+1}) = y(t_n)+\sum_{i=0}^N \F(t_{n+1-i},y^{(P)}_{n+1-i})\int_{t_n}^{t_{n+1}} L_{N-i}(\tau)d\tau
\end{equation}
\end{itemize}
For a four point approximation, without loss of generality, we may consider $t_{3-i}=-ih$. Thus we have,
\begin{eqnarray} 
L_0(t)&=&\frac{(t+h)(t+2h)(t+3h)}{(0+h)(0+2h)(0+3h)} = \frac{1}{6h^3}(t^3+6ht^2+11h^2t+6h^3) \nonumber \\
L_1(t)&=&\frac{(t-0)(t+2h)(t+3h)}{(-h+0)(-h+2h)(-h+3h)} = -\frac{1}{2h^3}(t^3+5ht^2+6h^2t) \nonumber \\
L_2(t)&=&\frac{(t-0)(t+h)(t+3h)}{(-2h+0)(-2h+h)(-2h+3h)} = \frac{1}{2h^3}(t^3+4ht^2+3h^2t) \nonumber \\
L_3(t)&=&\frac{(t-0)(t+h)(t+2h)}{(-3h+0)(-3h+h)(-3h+2h)} = -\frac{1}{6h^3}(t^3+3ht^2+2h^2t) \nonumber
\end{eqnarray}
 For the corrector,  the integration must be performed  in the domain of $(-h,0)$. Performing the integrations we have,
\begin{equation}
{_{-h}\I_0}\left[ L_0\right] = \frac{3h}{8}~,{_{-h}\I_0}\left[ L_1\right] = \frac{19h}{24}~,{_{-h}\I_0}\left[ L_2\right] = \frac{-5h}{24}~,{_{-h}\I_0}\left[ L_3\right] = \frac{h}{24} 
\end{equation}
%+55.0/+24.0, -59.0/+24.0, +37.0/+24, -9.0/+24.0
For the predictor the integration must performed in the domain $(0,h)$,
\begin{equation}
{_0\I_h}\left[ L_0\right] = \frac{55h}{24}~,{_0\I_h}\left[ L_1\right] = \frac{-59h}{24}~,{_0\I_h}\left[ L_2\right] = \frac{37h}{24}~,{_0\I_h}\left[ L_3\right] = \frac{h}{24} 
\end{equation}
\subsection{Runge-Cutta}
%
\section{Numerical Methods for solving Fractional Order Differential Equations(FODE)}
%
For this section we will exactly follow Deithelm approach. The suggested equation for dealing with practical applications is the following, 
\begin{equation}\label{fracde}
 ^{\bf C}_a\D_t^\alpha y(t) = \F(y,t)
\end{equation}
by fraction integrating the both sides of Eq.~\ref{fracde}, we have,
\begin{equation}\label{fode-numsol}
y(t) = \sum_0^{[\alpha]-1} \frac{t^{\nu}}{\nu!}y^{(\nu)}(0) + \frac{1}{\Gamma(\alpha)}\int_0^t \frac{\F(y(\tau),\tau)}{(t-\tau)^{\alpha-1}}d\tau
\end{equation}
The next step is to evaluate the integral in Eq.~\ref{fode-numsol}. Following Deithelm,
 one may slice the hole integration into slices which each slice could be evaluated
  by qudrature rule. In Deithelm they used the linear interpolation of integrand between two point in each slice.
  \begin{equation}
   \int_0^{t_{k+1}} (t-\tau)^{\alpha-1} \F(\tau)d\tau \simeq  \int_0^{t_{k+1}} (t-\tau)^{\alpha-1} \tilde{\F}(\tau)d\tau = \sum_{j=0}^{k+1} a_{j,k+1} \F(t_j)
  \end{equation}
with
\begin{equation}
 a_{j,k+1} =\frac{h^\alpha}{\alpha(\alpha+1)}
 \begin{cases}
      k^{\alpha+1} - (k-\alpha)(k+1)^\alpha & \text{if $j=0$}\\
       (k-j+2)^{\alpha+1} + (k-j)^{\alpha+1} -2 (k-j+1)^{\alpha+1} &  \text{if $1\preceq j \preceq k$}\\
       1 & \text{if $j=k+1$} \\
    \end{cases}
\end{equation}
The corrector for One step Adams-Moulton becomes,
\begin{equation}
y_{k+1} = \sum_0^{[\alpha]-1} \frac{t^{\nu}}{\nu!}y^{(\nu)}(0) + \frac{1}{\Gamma(\alpha)} \left [a_{k+1,k+1} \F(t_{k+1},y^{P}_{k+1}) + \sum_{j=0}^{k} a_{j,k+1} \F(t_j,y_j) \right]
\end{equation}
replacing the intgral in Eq.~\ref{fode-numsol} with rectangle rule will eliminat the share of the $y_{k+1}$, thus the equation will serve as a predictor estimation of the solution at $k+1$th point. In this case we have:
\begin{equation}
   \int_0^{t_{k+1}} (t-\tau)^{\alpha-1} \F(\tau)d\tau \simeq  \int_0^{t_{k+1}} (t-\tau)^{\alpha-1} \tilde{\F}(\tau)d\tau = \sum_{j=0}^{k+1} b_{j,k+1} \F(t_j)
\end{equation}
with $b_{j,k+1}=\frac{h^\alpha}{\alpha}((k+j-1)^\alpha-(k-j)^\alpha)$.
\begin{equation}
y^P_{k+1} = \sum_0^{[\alpha]-1} \frac{t^{\nu}}{\nu!}y^{(\nu)}(0) + \frac{1}{\Gamma(\alpha)}\sum_{j=0}^{k} b_{j,k+1} \F(t_j,y_j)
\end{equation}

  \subsection{Multi linear methods for the solutions of FODEs}
%
\subsection{Runge-Cutta}
%
\section{Numerical implementation}
%
\section{DC circuit analysis based on modified nodal node analysis(simple)}
%
\begin{figure}[h!]
  \begin{center}
\begin{circuitikz}[american]
    \path (0,0) coordinate(n) node[above,red]{$i$};
    % We want a star; let's use polar coordinates
    \draw(n) to[R, v =$R$, *-o] ++(-72:5) coordinate(c) node[left,red]{$j$};
    \draw(n) to[V, v<=$V$, *-o] ++(0:5)  coordinate(b) node[above,red]{$j$};
    \draw(n) to[C, v<=$C$, *-o] ++(72:5) coordinate(a) node[left,red]{$j$};
    \draw(n) to[I, v<=$I$, *-o] ++(144:5) coordinate(a) node[left,red]{$j$};
    \draw(n) to[L, v<=$L$, *-o] ++(215:5) coordinate(a) node[left,red]{$j$};
    % leads; the first one determines the horizontal shift (coordinate hh)
    % just change the ++(5,0) here and all will move logically
   % \draw (c) to[short, o-o] ++(5,0) coordinate(hh) node[red,right]{$C$};
    % now we use the syntax -| ( horizontal -| vertical) to draw the wires
    %\draw (b) to[short, o-o] (b -| hh) node[red,right]{$B$};
   % \draw (n) to[short, o-o] (n -| hh) node[red,right]{$N$};
    % a is a bit more complex because it's not straight on
   %S \draw (a) to[short, o-] ++(0,1) coordinate(aa) % small vertical wire
        %  to[short,-o] (aa -| hh) node[red,right]{$A$};
\end{circuitikz}
\end{center}
\caption{Schematic of a possible node configuration.}
\end{figure}
For the derivation NMA equations we consider the following book keepings:
\begin{itemize}
 \item  For each two nodes labeled by $i$ and $j$, we have $G_{ij}$, $I_{ij}$ and $V_{ij}$ corresponding to conductance, current and voltage difference between node $i$ and $j$, notice $I_{ij}=-I_{ji};~V_{ij}=-V_{ji}$.
 \item  $V_i$s correspond to the voltage difference between node $i$ and datum. The label $i=1 \cdots N$. Thus there are $N+1$ nodes.
 \item In each circuit, we have impedences, current sources and voltage sources.
  Thus the the unkown quantities are:
  \begin{enumerate}
  \item[(1)] node votages,
  \item[(2)] the currents corresponding to each voltage source.
  \end{enumerate}
\item Therefore, there are $N+M$ unkowns
   corresponding to the voltage of each node (N nodes) and the current for each
    voltage source ($M$ voltage sources).
\end{itemize}
For each node the following KCL equation holds:
\begin{equation}\label{eq0}
 \forall i \in\aleph: \sum_{s\in \Bbbk_i} V_{is}\G_{is} + \sum_{s\in \Bbbk_i} I^{(vs)}_{is} = -\sum_{s\in \Bbbk_i}  I^{(cs)}_{is}
\end{equation}
where the upper case $cs$ and $vs$ holds for current source and voltage source respectievly. The $\Bbbk_i$ means the set of all nodes which are
 connected to node $i$. Further rearranging:
\begin{eqnarray}
 \sum_{s\in \Bbbk_i} V_{is}\G_{is} &=& \sum_{s\in \Bbbk_i} (V_i-V_s)\G_{is} \nonumber \\
  &=&(\sum_{s\in \Bbbk_i} \G_{is})V_i - \sum_{s\in \Bbbk_i} V_s\G_{is}
\end{eqnarray}
one may define $\tilde{\G}_{ij}=\delta_{ij}\sum_{s\in \Bbbk_i}\G_{is}+(\delta_{ij}-1)\G_{ij}$
%
Thus Eq.~\ref{eq0} becomes:
\begin{equation}\label{eq1}
 \forall i \in\aleph: \sum_{s} V_s\tilde{\G}_{is} + \sum_{s\in \Bbbk_i} I^{(vs)}_{is} = -\sum_{s\in \Bbbk_i}  I^{(cs)}_{is}
\end{equation}
By inspecting Eq.~\ref{eq1} one may recognise that all unkowns are on left hand side, while the known quantities are on the right handside.
To further the MNA analysis one may define the unknown vector
${\bf X}=\begin{bmatrix}
    V     \\
    I
\end{bmatrix}$
  with
 ${\bf V}=\begin{bmatrix}
    V_1     \\
    V_2    \\
    \vdots \\
    V_{N}
    \end{bmatrix}$
    and
 ${\bf I} = \begin{bmatrix}
      I_1 \\
    I_2 \\
    \vdots \\
    I_M
\end{bmatrix}$.
Which is vector with $N+M$ elements.
While $V_i$s have a natural indexing related to node numbers,
one may realize a map between $I^{(vs)}_1\hdots I^{(vs)}_M$ and $I^{(vs)}_{ij}$
i.e. a map in the form of $(i,j)\rightarrow s$ with $s=1\hdots M$.
With such notation one may construct an ${\bf A}$ matrix which upon
 operation on, $I$, would produce the second term on the left hand of
 Eq.~\ref{eq1}. Such matrix only hold $-1,0,1$ elements. $-1$ corresponds to the current through negative terminal of voltage source,
  0 means there is no voltage source connected to this node, and $+1$ corresponds to positive part of voltage source.\\
  Moreover, the $A^T$ multipied to the $V$ vector would result in the
   $V_{is}=V_i-Vs$ which is a known quantity. On the other side of the
    Eq.\ref{eq1}, one may construct the vector ${\bf U}= \begin{bmatrix}
      I^{cs} \\
     V^{vs}
\end{bmatrix}$ with ${\bf I^{(cs)}} = \begin{bmatrix}
      I^{cs}_1 \\
    I^{cs}_2 \\
    \vdots \\
    I^{cs}_N
    \end{bmatrix}$ with $I^{cs}_i =-\sum_{s\in \Bbbk_i}  I^{(cs)}_{is}$
%
and vector $V^{vs}=\begin{bmatrix}
      V^{vs}_1 \\
    V^{vs}_2 \\
    \vdots \\
    V^{vs}_M
    \end{bmatrix}$, for $V^{vs}_s$ we have the same mapping
    $s \rightarrow (i,j) $ as of $I^{(vs)}_s$.
    In this regards, the set of equations may be writen as:
    \begin{eqnarray}
     A^T V &=& V^{vs} \nonumber \\
     \tilde{\G} V + A I &=& I^{(vs)}
    \end{eqnarray}
    Finally in the absense of voltage dependent current source
     one may write the final NMA equation as:
     ${\bf B X = U}$,
     with ${\bf B = \begin{bmatrix}
      \tilde{G} & A \\
            A^T & D
    \end{bmatrix}}$. With matrix $D$ is an $M\times M$ matrix with all of its elements equal to zero.

\section{Transient simulation of electrical circuits(passive parts only)}
%
For transient analysis of electrical circuits, one may exploit the
 static one with only current, voltage sources and resistors. To do this
  its is necessary to discritize the resulting equations wich consists
   of capacitors, inductances as well as CPE elements. Implementation of CPEs are the main goal of this documentation. After discritization of
    differential equations, one may collect the quantities of the former
     step as either current or voltage sources (depending on the formulation), the rest could be handled as a resistor with the resistance which depends on the $\Delta t$ and the constant of the
      particular elements. In the following we try to perform such procedure for different passive circuit elements.
\begin{itemize}
 \item Euler discritization for capacitors (corrector form):
 \begin{eqnarray}
c\frac{d V}{dt} = i&\Longrightarrow& c(V_{k+1}-V_k) = \frac{\Delta t}{2}(i_k+i_{k+1}) \nonumber \\
i_{k+1} &=& \frac{2c}{\Delta t} V_{k+1}  \underbrace{-\frac{2c}{\Delta t} V_{k}}_{I_{eq}}
 \end{eqnarray}
 the resulting equivalent circuit is:
 \begin{figure}[h!]
  \begin{center}
    \begin{circuitikz}
      \draw (0,0)
      to[V,v=$I_{eq}$] (0,2) % The voltage source
      to[short] (2,2)
      to[R=$\frac{2\Delta t}{c}$] (2,0) % The resistor
      to[short] (0,0);
    \end{circuitikz}
    \caption{Equivalent circuit for a capacitor.}
  \end{center}
\end{figure}
\item Euler discritization for inductors (corrector form):
 \begin{eqnarray}
-L\frac{d i}{dt} = V&\Longrightarrow& -L(i_{k+1}-i_k) = \frac{\Delta t}{2} (V_k+V_{k+1}) \nonumber \\
i_{k+1} &=&\underbrace{ -\frac{\Delta t}{2L} V_k + i_k}_{I_{eq}} - \frac{\Delta t}{2L}V_{k+1}
 \end{eqnarray}
 the resulting equivalent circuit is:
 \begin{figure}[h!]
  \begin{center}
    \begin{circuitikz}
      \draw (0,0)
      to[V,v=$I_{eq}$] (0,2) % The voltage source
      to[short] (2,2)
      to[R=$-\frac{2L}{\Delta t}$] (2,0) % The resistor
      to[short] (0,0);
    \end{circuitikz}
    \caption{Equivalent circuit for a inductor.}
  \end{center}
\end{figure}
\end{itemize}

\subsection{Simulation of parts without FO components}
%
\subsection{Simulation of parts with FO components}
%
\subsection{Implementation of transient algorithem}
%
\subsection{Inclusion of SPICE net lists}
\section{Some applications of FOC in the electrochemistry}

\section{Possible integration into QUCS}
%
\section{Higher order finite difference for integer order derivatives}
%
\section{Finite difference FOD}
%
\begin{acknowledgments}
\end{acknowledgments}
\section{Appendix}
 %%%%%%%%%%%%%%%%%%%%%%%%%%%%%%%%%%%%%%%%%%%%%%%%%%%%%%%
 %=========================================================================
 %%%%%%%%%%%%%%%%%%%%%%%%%%%%%%%%%%%%%%%%%%%%%%%%%%%%%%%
 \newpage
\nocite{apsrev41Control}
\bibliographystyle{apsrev4-1}
\bibliography{draft1.bib}

\end{document}



